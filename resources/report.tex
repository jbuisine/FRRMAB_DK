\documentclass{article}
\newcommand\tab[1][0.6cm]{\hspace*{#1}}
\usepackage[linesnumbered,algoruled,boxed,lined]{algorithm2e}
\usepackage[utf8]{inputenc}
\usepackage{fancyhdr}
\usepackage{natbib}
\usepackage{graphicx}
\usepackage{dirtytalk}
\usepackage[table]{xcolor}%

% Display format page configuration
\addtolength{\oddsidemargin}{-.875in}
\addtolength{\evensidemargin}{-.875in}
\addtolength{\textwidth}{1.75in}
\addtolength{\topmargin}{-.875in}
\addtolength{\textheight}{1.75in}

\newlength\mylen
\newcommand\myinput[1]{%
  \settowidth\mylen{\KwIn{}}%
  \setlength\hangindent{\mylen}%
  \hspace*{\mylen}#1\\}

%Setting color syntax language%
\usepackage{listings}
\usepackage{color}
 
\definecolor{codegreen}{rgb}{0,0.6,0}
\definecolor{codegray}{rgb}{0.5,0.5,0.5}
\definecolor{codepurple}{rgb}{0.58,0,0.82}
\definecolor{backcolour}{rgb}{0.95,0.95,0.92}
\definecolor{keywordcolor}{rgb}{1,0.20,0.1}


 
\lstdefinestyle{mystyle}{
    backgroundcolor=\color{backcolour},   
    commentstyle=\color{codegreen},
    keywordstyle=\color{keywordcolor},
    numberstyle=\tiny\color{codegray},
    stringstyle=\color{codepurple},
    basicstyle=\footnotesize,
    breakatwhitespace=false,         
    breaklines=true,                 
    captionpos=b,                    
    keepspaces=true,                 
    numbers=left,                    
    numbersep=5pt,                  
    showspaces=false,                
    showstringspaces=false,
    showtabs=false,                  
    tabsize=2
}
 
\lstset{style=mystyle}

% Settings head and footer
\pagestyle{fancy}
\fancyhf{}
\rhead{June 2017}
\lhead{FRRMAB using Directions Knowledge}
\cfoot{\thepage}

% Title configuration
\title{\textbf{Adaptive Operators Selection into MOEA/D using directions knowledge}}
\author{F. Teytaud, S. Verel, J. Buisine}
\date{}

\begin{document}

\maketitle

\vspace{30mm}

\tableofcontents
\newpage

\section{Abstract}

This paper presents new approach of \textit{Adaptive Operator Selection} (AOS) into \textit{Multiobjective Evolutionary Algorithm based on Decomposition} (MOEA/D). MOEA/D decomposes a multiobjective optimization problem into a number of scalar optimization subproblems and optimizes them simultaneously. AOS' aim is to select the best operator at iteration $i$ based on their recent performances within an optimization process. 

\vspace{3mm}

The AOS algorithm structure is mainly based on the \textit{Fitness-Rate-Rank-based Multi-Armed Bandit} (FRRMAB) introduced in \citep{DBLP:journals/tec/LiFKZ14}. FRRMAB uses Upper Confidence Bound algorithm to choose the best operator at iteration $i$. In our article, we introduced the \textit{Fitness-Rate-Rank-based Multi-Armed Bandit using Directions Knowledge} (FRRMABDK) algorithm. Its aim is to take care of closest subproblems operators rewards information to select the best operator. 


\newpage

\section{Introduction}

MOEA/D decomposes a multiobjective optimization problem into a number of scalar optimization subproblems and optimizes them simultaneously during optimization process. 

\newpage
\section{FRRMAB using Directions Knowledge}


\subsection{Updates of Fitness Rate Ranks}


\begin{algorithm}[H]
\SetAlgoLined

Initialize each reward $Reward_i$ = 0; \\
Initialize each $nb_{op}$ = 0;

\For{$i := 0$ to SlidingWindow[$subProblem$].$length$}{
        $op$ = SlidingWindow[$subProblem$].$getIndexOp(i)$; \\
        $FIR$ = SlidingWindow[$subProblem$].$getFIR(i)$; \\ 
        $Reward_{op}$ = $Reward_{op} + FIR$; \\ 
        $nb_{op}++$;
}

\For{$n$ : $neighbors$}{
    
    $dist$ = $|subProblem - n |$;

    \For{$j := 0$ to SlidingWindow[$n$].$length$}{
        $op$ = SlidingWindow[$n$].$getIndexOp(j)$; \\
        $FIR_n$ = SlidingWindow[$n$].$getFIR(j)$; \\ 
        $Reward_{op}$ = $Reward_{op} + (FIR_n * \alpha^{dist})$; \\ 
    }
}

Rank $Reward_i$ in descending order and set $Rank_i$ to be the rank value of operator $i$; \\

\For{$op := 1$ to $K$}{
    $Decay_{op} = D^{Rank_{op}} * Reward_{op}$;
}

$DecaySum = \sum_{op=1}^{K} Decay_{op}$;

\For{$op := 1$ to $K$}{
    $FRR_{op} = Decay_{op} / DecaySum$;
}

\caption{updateCreditAssignmentSubProblem($subProblem$)}
\end{algorithm}

\subsection{UCB algorithm, operator choice}

\vspace{3mm}

\begin{algorithm}[H]
\SetAlgoLined

$FRR$ = $FRRs(subProblem)$; \\

\If{There are operators that have not been selected}{
    $op_t$ = one op randomly choose from operators pool not already chosen;
}\Else{
    $op_t$ = $argmax_{i=\{i...K\}} \left( FFR_{i,t} + C * \sqrt{\frac{2 * ln\sum_{j=1}^{K} nb_{j,t}}{nb_{i,t}}} \right)$
}
\caption{FRRMABDK($subProblem$)}
\end{algorithm}


\subsection{MOEA/D using FRRMABDK}

\tab The Multiobjective Evolutionary Algorithm Based on Decomposition \citep{DBLP:journals/tec/ZhangL07} is a scalarizing approach with population. Scalarizing approaches consist in transforming the original problem into a single-objective one by means of an aggregation of the objective function values. There is two typical scalarizing functions used into this paper. 

\begin{itemize}
    \item
        Weighted-sum scalarizing function \citep{DBLP:conf/emo/LiefoogheDVAT17} defined below.
        
        $$g_\lambda(x) = \lambda_1.f_1(x) + \lambda_2.f_2(x)$$
        where $x \in S_n$ is a candidate solution, and $\lambda = (\lambda_1, \lambda_2)$ is a weighting coefficient vector. 
    \item 
        Tchebycheff scalarizing function defined as follows.
        
        $$g_{\lambda} (x) = \min \bigg\{ \lambda_1 * |f_1(x) - r_1|, \lambda_2 * |f_2(x) - r_2| \bigg\} $$
        
        where $r$ is a reference point in the objective space, as example $r(0,0)$. 
\end{itemize}

\vspace{3mm}

\begin{algorithm}[H]
\SetAlgoLined

Initialize the population $\{x^1, ... , x^N\}$ and $z*$; \\
Initialize $\alpha$ [0, 1]; 

\For{$i := 0$ to $N$}{

    $B(i) = \{i_1, ... , i_T\}$ where $\lambda^{i_1}, ... , \lambda^{i_T}$ are the $T$ closest weight vectors to direction $\lambda^i$;
    
    Initialize sliding window of size $W$ for direction $i$. 
}

\Repeat{\textit{Stopping criterion is not satisfied}}
{
    \For{$i := 0$ to $N$}{
        
        
        $op$ = $FRRMABDK(i)$; \\
        
        $x^i$ = $pop[i]$; \\ 
        generate $y$, a mutant solution from $x^i$ using $op$ operator; \\
        $repair(i)$; 
        
        $neighbors$ = $getNeighbors(i)$; \\
        
        $FIR_{op} = 0$; 
        
        \For{$n$ : $neighbors$}{
            
            $\nabla = \frac{g\left(x^n | \lambda^n, z* \right) - g\left( y | \lambda^n, z* \right)}{g\left(x^n | \lambda^n, z* \right)} $; 
            
            \If{$\nabla > 0$}{
                Replace $x^n$ with $y$; \\
                $FIR_{op} = FIR_{op} + \nabla$; 
            }
        }
        
        $updateSlidingWindow(i, op, FIR_{op})$; \\
        $updateCreditAssignmentSubProblem(i)$; 
    }
}
\caption{MOEA/D-FRRMABDK}
\end{algorithm}

\newpage
\section{Test suites}
\subsection{QAP problem instance}

\subsection{Operators pool}
List of these operators :
\begin{enumerate}
    \item Mutation operator which randomly choose $x_1$, an index of the QAP solution and permute it with the next cell. 
    \item Mutation operator which randomly permutes $x_1$ and $x_2$, two different indexes of the solution.
    \item Mutation operator which randomly permute two times $x_1$ and $x_2$ which are two different indexes of the solution.
\end{enumerate}


\subsection{Results}

\tab Same global parameters are used into these test results :

\begin{center}
    \begin{tabular}{|l|c|r|}
      \hline
       seed & 10 \\
      \hline
       mu & 100 \\
      \hline
       T & 20 \\
      \hline
       W & 15 \\
      \hline
       C & $\sqrt{2}$ \\
      \hline
       D & 0.5 \\
      \hline
       evaluation & 1000000 \\
      \hline
    \end{tabular}
\end{center}

Note that, the Weighted-sum scalarizing function is used for each test suites.

\vspace{3mm}

\tab We compare the \textit{\textbf{hypervolume}} metric obtained by the approximation of pareto front from FRRMAB and FRRMABDK. Operators list is initialized using different order combinations. 

\vspace{3mm}
First column defines operators combination, the second FRRMAB results and the others, FRRMABDK results with different \textbf{$\alpha$} affinity parameter value.
  
\begin{center}
    \begin{tabular}{|l|c|r|r|r|r|r|}
      \hline
      Operators & FRRMAB & ...DK, $\alpha = 0.5$ & ...DK, $\alpha = 0.6$ & ...DK, $\alpha = 0.7$ & ...DK, $\alpha = 0.8$ & ...DK, $\alpha = 0.9$\\
      \hline
      (1, 2, 3) & 8.96529e+11 
                & \cellcolor{green!20}8.9417e+11 
                & \cellcolor{green!50}8.68122e+11 
                & \cellcolor{green!100}8.53514e+11 
                & \cellcolor{green!20}8.83109e+11 
                & \cellcolor{green!50}8.66716e+11\\
                
      (1, 3, 2) & 8.69997e+11 
                & \cellcolor{green!70}8.33684e+11 
                & \cellcolor{red!40}8.93059e+11 
                & \cellcolor{red!20}8.83973e+11 
                & \cellcolor{red!50}9.14678e+11 
                & \cellcolor{red!50}9.17512e+11\\
                
      (2, 1, 3) & 8.47605e+11 
                & \cellcolor{red!20}8.62056e+11 
                & \cellcolor{green!80}7.80205e+11 
                & \cellcolor{red!40}8.89187e+11 
                & \cellcolor{red!30}8.5989e+11 
                & \cellcolor{red!35}8.63313e+11\\
                
      (2, 3, 1) & 8.47605e+11 
                & \cellcolor{green!30}8.15611e+11 
                & \cellcolor{red!10}8.5119e+11 
                & \cellcolor{red!50}8.90212e+11 
                & \cellcolor{red!70}9.15997e+11 
                & \cellcolor{red!70}9.10662e+11\\
                
      (3, 1, 2) & 8.66909e+11 
                & \cellcolor{red!20}8.90197e+11 
                & \cellcolor{red!20}8.9376e+11 
                & \cellcolor{green!50}8.31216e+11 
                & \cellcolor{red!20}8.9186e+11 
                & \cellcolor{green!80}7.90285e+11\\
                
      (3, 2, 1) & 8.88218e+11 
                & \cellcolor{green!50}8.36726e+11 
                & \cellcolor{green!50}8.38885e+11 
                & \cellcolor{red!15}8.94274e+11 
                & \cellcolor{green!10}8.79118e+11 
                & \cellcolor{red!25}9.07816e+11\\
      \hline
    \end{tabular}
\end{center}

\newpage
\section{Conclusion}

\tab Others operators... change MOEA/D $y$ solution generation...


\bibliographystyle{abbrv}
\bibliography{references}
\end{document}
